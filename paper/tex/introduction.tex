% !TEX encoding = UTF-8 Unicode


<< Le Bachelor à HEC Lausanne est très quantitatif >>, une rumeur maintes fois entendue à l’intérieur comme à l’extérieur des murs du campus lémanique, notamment lors de comparaisons avec d’autres établissements suisses. Cette « vérité populaire » a été le point de départ de notre étude, nous donnant l’envie de dépasser le stade du questionnement initial pour tenter de vérifier ou d’infirmer cet adage au travers d’une analyse empirique.

Au-delà de la simple rumeur précitée, se pose la question de l’adaptation du cursus aux connaissances préalables des étudiants, ainsi que sa pertinence par rapport à l’orientation future et aux débouchés souhaités par ces derniers. À l’époque de nos parents, la filière économique était divisée en plusieurs voies dès la sortie du propédeutique, aujourd’hui réunies au sein d’un même tronc commun (obligatoire) de deux ans. Face à ces plans d'études, nombreux sont les étudiants se questionnant sur le bien-fondé de l’obligation de suivre certains enseignements qu’ils n’appliqueront jamais : marketing pour les futurs <<financiers>>, finance pour les futurs « marketeurs », et bien d’autres paires de matières parfois perçues comme peu complémentaires. Cette volonté de choisir une voie spécifique dès le début du cursus semble d’ailleurs être en ligne avec la tendance générale du monde de l’emploi, à l’heure de la spécialisation des compétences.

Plus précisément, nous allons nous intéresser à l’effet de la performance excédentaire (ajustée par rapport à la performance générale) dans les cours à composante quantitative de première année sur la moyenne générale de deuxième, puis de troisième année. Nous allons également analyser l’effet d’être meilleur dans les matières quantitatives sur la probabilité de succès au Bachelor. En complément, nous décomposerons l’effet de chacun des cours individuels de première année sur la moyenne générale de deuxième année.

Notre analyse montre que les étudiants meilleurs dans les cours quantitatifs de première année, relativement à leur performance moyenne, obtiennent de meilleurs résultats généraux en deuxième année. Par contre, cet effet s’estompe en troisième année, perdant toute significativité. Aussi, les élèves se situant dans les quantiles supérieurs de l’échantillon en termes de performance excédentaire dans les cours quantitatifs de première année ont une probabilité de succès au Bachelor très élevée (76\%), cette dernière étant très faible dans le cas contraire (28\%). Enfin, nous identifions des résultats opposés pour l’effet de la surperformance dans les cours du premier et du deuxième semestre de première année, respectivement négatifs et positifs (ou nuls), sur la performance de deuxième année, ceci indépendamment du type d’enseignement considéré.


La littérature existante sur l'apprentissage traite de deux aspects importants pour notre question de recherche. Plusieurs auteurs se sont intéressés à l’impact du savoir antérieur sur la performance, ainsi qu’aux facteurs influençant la réussite académique sur le plan général, deux points directement liés à notre travail.

Premièrement, il en ressort que le savoir antérieur produit deux effets opposés sur l’apprentissage, au point de parler d’un paradoxe du savoir antérieur (\cite{roschelle}). Ce dernier résulte d’un côté de la nécessité du savoir antérieur pour la compréhension future, de l’autre du fait que l’apprenant ait tendance à construire des interprétations du savoir nouveau concordant avec le savoir antérieur, ce qui peut mener à une contradiction. Il se manifeste notamment par des effets de confusion ou des erreurs d’interprétation. \cite{champagne} montrent que l’apprentissage réel consiste en une intégration du savoir nouveau dans celui acquis préalablement, tel un changement conceptuel, plutôt qu'en un processus indépendant du savoir antérieur.

Deuxièmement, parmi les facteurs influençant la réussite académique, \cite{touron} en distingue deux principaux: les caractéristiques de l'institution, telles que la qualité et le style d'enseignement, et les caractéristiques personnelles de l’étudiant-e, dont la performance académique antérieure. De plus, \cite{garcia} ont récemment souligné à nouveau la forte corrélation de cette dernière avec la réussite à l'université.

Notre travail s'appuie sur la littérature précitée et y présente une extension, dans la mesure où il traite pour la première fois de ces questions dans le contexte précis des études économiques, plus particulièrement dans le cadre du Bachelor d'HEC Lausanne. Cette extension est importante car elle remet en question le statu quo du système actuel et révèle de potentielles sources d’amélioration des plans d’études. Elle constitue également une base pour la comparaison avec d’autres établissements du pays.


Dans cet article, nous allons d'abord présenter les données utilisées ainsi que le travail de préparation réalisé, y compris la création de différentes variables d'intérêt. Nous poursuivrons avec quelques statistiques descriptives de notre échantillon d'étudiants, discuterons des biais potentiels auxquels nous nous sommes confrontés, puis expliquerons les modèles économétriques utilisés afin de répondre à nos questions de recherche. Enfin, nous présenterons et discuterons nos résultats d'analyses principaux et complémentaires.
