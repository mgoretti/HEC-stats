
Dans le cadre de notre travail, nous souhaitions analyser différents facteurs de réussite au Bachelor d'HEC Lausanne. Nous avons notamment voulu vérifier la véracité de certaines idées reçues au sujet du cursus lausannois, telles que son orientation réputée très quantitative, la rumeur allant jusqu'à dire qu'il est impossible de réussir son cursus en étant mauvais en mathématiques.

Nos analyses démontrent en effet que les étudiants réussissant mieux les cours quantitatifs de première année, comparativement aux autres types de cours, auront une probabilité de réussite supérieure au Bachelor, ainsi qu'une meilleure moyenne en deuxième année. Un étudiant surperformant dans les cours quantitatifs au détriment des cours opposés (en fixant MIX) aura près de 76\% de chances de réussir son Bachelor (à l'extrême de la distribution, delta = +0.1) et une moyenne espérée de 4.32 en deuxième année. À l'inverse, une personne située à l'autre extrême (delta = -0.1) n'aura que 28\% de chances de réussir son Bachelor et une moyenne espérée de 4.06, correspondant à un ranking inférieur de 10\%. Ces résultats montrent également à quel point le système de libre choix des cours en troisième année permet aux étudiants de s'orienter vers leurs types d'enseignements de prédilection, le fait d'être meilleur dans les cours quantitatifs n'influençant plus la moyenne générale de cette dernière année.

Des analyses complémentaires et auxiliaires révèlent que d'autres facteurs, tels que le type de diplôme d'accès obtenu, peuvent également impacter la réussite future des étudiants, tout en confirmant l'importance des compétences mathématiques dans cette dernière. En effet, le diplôme d'accès donnant la probabilité de réussite future la plus élevée n'est autre que la maturité \textit{maths-physique} suisse.

Nos résultats confirment nos hypothèses initiales et les croyances générales des étudiants sondés dans le cadre de ce travail. Les étudiants ayant réussi leur première année, donc ayant eu un aperçu des enseignements et des compétences demandées, semblent avoir une vision de notre problématique qui concorde avec la réalité. Nous partons du principe qu'ils acceptent la situation tant qu’ils réussissent leur cursus, mais y sont-ils réellement favorables, ou souhaiteraient-ils du changement ?

Justement, au travers de la comparaison des effets de nos variables explicatives sur les moyennes générales de deuxième et troisième année, nous voyons que le passage à un système de libre choix des cours semble plus équitable pour les étudiants par rapport à leurs compétences propres. Dès lors, il conviendrait de réfléchir sur l’extension de ce système aux autres années du Bachelor. D’ailleurs, en automne 2014, le Décanat de la Faculté a pris la décision d’offrir une liberté, certes restreinte\footnote{Le nombre de cours disponibles, à peine plus élevé que le minimum requis pour valider l'année, ainsi qu'une contrainte sur la répartition en termes de crédits ECTS entre les deux semestres, ne permettent qu'une flexibilité très limitée aux étudiants}, dans le choix des cours de deuxième année – une première depuis l'introduction du Système de Bologne.

De plus, à l’époque de nos parents, les cursus de Licence HEC étaient séparés dès le début en fonction de l’orientation future souhaitée et, hormis l'année propédeutique, le reste des plans d'études était basé sur un système de libre choix des enseignements. Les résultats de nos analyses semblent suggérer qu’en cas de séparation initiale des plans d’études par domaine spécifique, les étudiants seraient globalement plus à même de réussir. En effet, nous observons que les élèves ayant des difficultés en deuxième année en raison de leurs lacunes dans les cours quantitatifs réussissent très bien en troisième. Ainsi, nous pouvons nous demander si cette réussite est due à la facilité des enseignements choisis en dernière année ou à l'inutilité des compétences quantitatives, requises en début de cursus, dans ces derniers - et au-delà, dans le monde professionnel y associé. Faudrait-il alors envisager un <<retour aux sources>> pour être plus équitable envers les étudiants se destinant aux domaines tels que le marketing, management et droit?

Pour conclure, notre travail ayant été réalisé dans le contexte spécifique de l’Université de Lausanne, et de par la diversité des plans d’études entre les établissements, nos résultats sont difficilement généralisables au-delà. La méthodologie appliquée pourrait toutefois être reprise dans le cadre d’une extension de notre analyse à d’autres universités suisses, dans le but d’effectuer des comparaisons entre les cursus de Bachelor, au travers desquelles nous pourrions infirmer ou confirmer l’orientation plus « quantitative » du plan d’études lausannois. Cette analyse permettrait également de faire ressortir des différences intéressantes pour de futurs étudiants lors du choix de leur filière d’études, en fonction de leurs connaissances préalables et autres affinités. Enfin, nous évoquions également la décision récente du Décanat concernant l’introduction d’un système de choix de cours en deuxième année. Ce choix, bien que restreint, semble déjà avoir amélioré la motivation et la performance des étudiants, comme plusieurs professeurs nous l’ont affirmé. Il serait donc intéressant d’analyser à nouveau, dans quelques années, l’impact de ce changement sur nos paramètres d’intérêt.
