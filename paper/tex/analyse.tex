
\subsection{Statistiques descriptives de l'échantillon}

Notre échantillon se compose de 2460 étudiants ayant terminé leur plan d'études entre 2010 et 2014, échecs et réussites confondus (tableau \ref{tab:sex}), avec une proportion de femmes restée stable dans le temps, toujours proche de sa moyenne globale (35.28\%). Notons que les individus ayant commencé leur plan d'études après 2011 ont tous échoué ou abandonné, étant donné qu'ils n'auraient pas pu terminer leur plan d'études autrement d'ici 2014. Ceci se confirme en comparant les tableaux \ref{tab:sex} et \ref{tab:echec}. D'ailleurs, nous pouvons observer (graphique \ref{fig:moyenne} ou tableaux \ref{tab:moyenne}, \ref{tab:moyenneFail} et \ref{tab:moyenneFini}) que la moyenne générale de l'ensemble des étudiants diminue entre 2007 et 2011, alors qu'elle augmente en ne considérant que les cursus réussis, suggérant une augmentation du nombre d'échecs, confirmée dans les faits (voir tableaux \ref{tab:sex} et \ref{tab:echec}, dans lesquels on observe une augmentation du nombre d'échecs en proportion du nombre total d'étudiants).

La carte \ref{fig:canton} permet de comparer la performance moyenne des deux côtés du \textit{Röstigraben} chez les étudiants gradués. Elle montre qu'après élimination des cantons sous-représentés de l'échantillon (moins de 3 observations), les étudiants bernois compensent la performance inférieure des zurichois pour aboutir à une moyenne comparable avec la Suisse Romande, tandis que les étudiants tessinois obtiennent de moins bons résultats.

Enfin, la distribution des variables delta individuelles et par catégories est illustrée par le graphique \ref{fig:deltas}, les détails des distributions par années par le tableau \ref{tab:deltas}. Elle est toujours centrée et l'écart-type ne varie que peu au sein de la même catégorie. Cependant, il est important de remarquer que les catégories ne contiennent pas le même nombre de crédits. Ainsi la plus grande (quantitative) présente un écart-type d'environ 5\%, alors que les deux autres se situent aux alentours de 17\%. Dans les régressions que nous mènerons, nous utiliserons les intervalles de confiance de deux écarts-types pour l'interprétation de l'influence de leurs coefficients.


\subsection{Discussion des biais potentiels}
 
Si nous avions simplement régressé les résultats de deuxième année sur la performance de première dans les différentes catégories de cours, les effets auraient tous été fortement significatifs et positifs, car incluant les caractéristiques intrinsèques de chaque individu telles que son aptitude à performer, particulièrement importante pour expliquer les notes obtenues. Au travers de l'utilisation des variables "Delta", nous pouvons résoudre ce problème de biais de variables omises. En effet, par construction, nos variables d’écart de performance suppriment l'effet des caractéristiques individuelles inobservées constantes dans le temps, telles que le fait d’être intrinsèquement « bon ». Ainsi, au niveau des cours individuels, les variables "Delta" décrivent exactement ce que nous souhaitions: le fait d'être meilleur dans un cours donné par rapport à sa performance générale de première année.

Aussi, notons qu'au niveau du travail sur nos données, nous avons procédé à une attribution arbitraire des enseignements aux catégories définies, pouvant potentiellement causer un biais dans nos résultats. Toutefois, nous estimons que notre allocation reflète la réalité de manière satisfaisante, ayant nous-mêmes suivi la plupart des cours recensés ou pu obtenir suffisamment de renseignements pour les attribuer pertinemment à une catégorie.

Nous étions également confrontés à un biais potentiel d'attrition, en raison de la perte d'individus au fil des années. Les étudiants qui échouent en deuxième et troisième année, ainsi que ceux qui ne réussissent pas leur première année, comptent malgré tout dans le calcul des quantiles pour ceux qui réussiront par après. Toutefois, ce phénomène ne constitue pas un problème dans notre cas, car les individus qui échouent ne font que déplacer la valeur des quantiles, ce qui n'intervient pas dans la différence calculée pour les deltas.

Enfin, dans nos analyses, nous utilisons les résultats de première année pour prédire ceux des années suivantes. En raison de la causalité temporelle, nous évitons tout problème d'endogénéité.

